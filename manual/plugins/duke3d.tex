\subsection{Duke3D}

\screenshot{plugins/images/ss-duke3d}{``Hollywood Holocaust'', the
  first level of Duke Nukem 3D}{fig:Duke3D}

This is a port of Duke Nukem 3D, derived from Fabien Sanglard's
Chocolate Duke.

\subsubsection{Installation}
The \fname{.GRP} and \fname{.CON} files from an original installation
of the game must be placed in the \fname{/.rockbox/duke3d/} on your
device. The shareware files work as well, and are available at
\wikilink{PluginDuke3D}.

\subsubsection{Music}
In-game music will not work by default. For it to work, you must
install a modified Timidity patchset in the
\fname{/.rockbox/timidity/} directory on your device. There should be
a \fname{/.rockbox/timidity/timidity.cfg} file located in this
directory, along with the instrument files. You must edit the
\fname{.cfg} file so that all the path names are absolute (i.e. in the
form \fname{/.rockbox/timidity/instruments/*.cfg}).

As above, there is a free patchset available from
\wikilink{PluginDuke3D}.

\subsubsection{Video}
Rotation of the video output is possible by choosing the correct video
option in the in-game menu. If your device's display is normally
320x240, for example, choosing the 240x320 option will rotate the
screen 90 degrees, and the keymap will be updated as well.

\subsubsection{Caveats}
Sound effects, enabled by default, could have a detrimental effect on
playability on some devices. If you notice excessive lag, try
disabling sound.

The default button mapping may not be optimal for gameplay. Set a
different mapping in the ``Keyboard'' section of the game setup. Note
that not all keys are mappable on all devices.

Some devices will low memory or large GRP files will prevent the game
from completely caching the GRP file in RAM, which could lead to disk
reads during gameplay. This might cause the game to lag slightly when
it happens, which is normal. The game should resume in a second or so.

On other devices, large GRP files will render the game completely
unplayable. If this occurs, try using the smaller shareware GRP. If
this still fails, then please see Appendix \ref{sec:feedback} for
instructions to report a bug.
