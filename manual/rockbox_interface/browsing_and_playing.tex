% $Id$ %
\chapter{Browsing and playing}
\section{\label{ref:file_browser}File Browser}
\screenshot{rockbox_interface/images/ss-file-browser}{The file browser}{}
Rockbox lets you browse your music in either of two ways. The
\setting{File Browser} lets you navigate through the files and directories on
your \dap, entering directories and executing the default action on each file.
To help differentiate files, each file format is displayed with an icon.

The \setting{Database Browser}, on the other hand, allows you to navigate
through the music on your player using categories like album, artist, genre,
etc.

You can select whether to browse using the \setting{File Browser} or the
\setting{Database Browser} by selecting either \setting{Files} or
\setting{Database} in the \setting{Main Menu}.
If you choose the \setting{File Browser}, the \setting{Show Files} setting
lets you select what types of files you wish to view. See
\reference{ref:ShowFiles} for more information on the \setting{Show Files}
setting.

\note{The \setting{File Browser} allows you to manipulate your files in ways
that are not available within the \setting{Database Browser}. Read more about
\setting{Database} in \reference{ref:database}. The remainder of this section
deals with the \setting{File Browser}.}

\opt{iriverh10,iriverh10_5gb}{\note{
If your \dap{} is a MTP model, the Music directory where all your music is stored
may be hidden in the \setting{File Browser}. This may be fixed by either
changing its properties (on a computer) to not hidden, or by changing
the \setting{Show Files} setting to all.
}}

\subsection{\label{ref:controls}File Browser Controls}
\begin{btnmap}
      \ActionStdPrev{}/\ActionStdNext{}
      \opt{HAVEREMOTEKEYMAP}{& \ActionRCStdPrev{}/\ActionRCStdNext{}}
         & Go to previous/next item in list. If you are on the first/last
           entry, the cursor will wrap to the last/first entry.\\
      %
      \opt{IRIVER_H100_PAD,IRIVER_H300_PAD}
        {
          \ButtonOn+\ButtonUp{}/ \ButtonDown
          \opt{HAVEREMOTEKEYMAP}{&
            \opt{IRIVER_RC_H100_PAD}{\ButtonRCSource{}/ \ButtonRCBitrate}
          }
          & Move one page up/down in the list.\\
        }
      \opt{IRIVER_H10_PAD}
        {
          \ButtonRew{}/ \ButtonFF
          & Move one page up/down in the list.\\
        }
      %
      \ActionTreeParentDirectory
      \opt{HAVEREMOTEKEYMAP}{& \ActionRCTreeParentDirectory}
      & Go to the parent directory.\\
      %
      \ActionTreeEnter
      \opt{HAVEREMOTEKEYMAP}{& \ActionRCTreeEnter}
      & Execute the default action on the selected file or enter a
        directory.\\
      %
      \ActionTreeWps
      \opt{HAVEREMOTEKEYMAP}{& \ActionRCTreeWps}
         & If there is an audio file playing, return to the
         \setting{While Playing Screen} (WPS) without stopping playback.\\
      %
      \nopt{player,SANSA_C200_PAD,erosqnative}%
        {%
          \ActionTreeStop
          \opt{HAVEREMOTEKEYMAP}{& \ActionRCTreeStop}
          & Stop audio playback.\\%
        }%
      %
      \ActionStdContext{}
      \opt{HAVEREMOTEKEYMAP}{& \ActionRCStdContext}
      & Enter the \setting{Context Menu}.\\
      %
      \ActionStdMenu{}
      \opt{HAVEREMOTEKEYMAP}{& \ActionRCStdMenu}
      & Enter the \setting{Main Menu}.\\
      %
      \nopt{erosqnative}{
        \opt{quickscreen}{
          \ActionStdQuickScreen
          \opt{HAVEREMOTEKEYMAP}{& \ActionRCStdQuickScreen}
          & Switch to the \setting{Quick Screen}
          (see \reference{ref:QuickScreen}). \\
        }
      }
      %
      \opt{SANSA_E200_PAD}{
        \ActionStdRec & Switch to the \setting{Recording Screen}.\\
      %
      }
      \nopt{touchscreen}{\opt{hotkey}{
        \ActionTreeHotkey
            &
        \opt{HAVEREMOTEKEYMAP}{
            &}
        Activate the \setting{Hotkey} function
        (see \reference{ref:Hotkeys}).
            \\
      }}
\end{btnmap}

\subsection{\label{ref:Contextmenu}\label{ref:PartIISectionFM}Context Menu}
\screenshot{rockbox_interface/images/ss-context-menu}{The Context Menu}{}

The \setting{Context Menu} allows you to perform certain operations on files or
directories.  To access the \setting{Context Menu}, position the selector over a file
or directory and access the context menu with \ActionStdContext{}.\\

\note{The \setting{Context Menu} is a context sensitive menu.  If the
\setting{Context Menu} is invoked on a file, it will display options available
for files.  If the \setting{Context Menu} is invoked on a directory,
it will display options for directories.\\}

The \setting{Context Menu} contains the following options (unless otherwise noted,
each option pertains both to files and directories):

\begin{description}
\item [View.]
  Displays the contents of the selected playlist file.
\item [Playing Next...]
  Enters the \setting{Playing Next Submenu} (see \reference{ref:playingnext_submenu}).
\item [Add to Playlist...]
  Enters the \setting{Add to Playlist Submenu} (see
  \reference{ref:addtoplaylist_submenu}).
\item [Rename.]
  This function lets the user modify the name of a file or directory.
\item [Cut.]
  Copies the name of the currently selected file or directory to the clipboard
  and marks it to be `cut'.
\item [Copy.]
  Copies the name of the currently selected file or directory to the clipboard
  and marks it to be `copied'.
\item [Paste.]
  Only visible if a file or directory name is on the clipboard. When selected
  it will move or copy the clipboard to the current directory.
\item [Delete.]
  Deletes the currently selected file. This option applies only to files, and
  not to directories. Rockbox will ask for confirmation before deleting a file.
  Press \ActionYesNoAccept{}
  to confirm deletion or any other key to cancel.
\item [Delete Directory.]
  Deletes the currently selected directory and all of the files and subdirectories
  it may contain. Deleted directories cannot be recovered. Use this feature with
  caution!
\item [Open with.]
  Runs a viewer plugin on the file. Normally, when a file is selected in Rockbox,
  Rockbox automatically detects the file type and runs the appropriate plugin.
  The \setting{Open With} function can be used to override the default action and
  select a viewer by hand.
  For example, this function can be used to view a text file
  even if the file has a non-standard extension (i.e., the file has an extension
  of something other than \fname{.txt}). See \reference{ref:Viewersplugins}
  for more details on viewers.
\item [Create Directory.]
  Create a new directory in the current directory on the disk.
\item [Properties / Show Track Info.]
  Shows properties such as size and the time and date of the last modification
  for the selected file. If used on a directory, the number of files and
  subdirectories will be shown, as well as the total size. If used on a supported
  audio file, its metadata will be displayed.
\opt{lcd_non-mono}{
\item [Set As Backdrop.]
  Set the selected \fname{bmp} file as background image. The bitmaps need to meet the
  conditions explained in \reference{ref:LoadingBackdrops}.
}
\item [Add to Shortcuts.]
  Adds a link to the selected item in the \fname{shortcuts.txt} file
  (see \reference{ref:MainMenuShortcuts}).
  If the file does not already exist it will be created in your Rockbox directory.
  Note that if you create a shortcut to a file, Rockbox will not open it upon
  selecting, but simply bring you to its location in the \setting{File Browser}.
\end{description}

\subsubsection{Set As...}
\begin{description}
  \item [Playlist Directory.]
     Set as default directory for the Playlist Catalogue.
  \opt{tagcache}{
  \item [Database Directory.]
     Rockbox usually stores database files in the \fname{/.rockbox} folder.
     You can choose another location for the database using this setting.
     This is mainly useful for multiboot targets, so the same database can
     be shared among several builds without needing to rebuild it each time.
  }
  \opt{recording}{
    \item [Recording Directory.]
     Save recordings in the selected directory.
  }
  \item [\label{ref:StartDirectory}Start Directory.]
   Set as default start directory for the File Browser.
   \note{If you have \setting{Auto-Change Directory} and
   \setting{Constrain Auto-Change} enabled, the directories returned will
   be constrained to the directory you have chosen here and those below it.
   See \reference{ref:ConstrainAutoChange}}
\end{description}

\subsection{\label{sec:virtual_keyboard}Virtual Keyboard}
\screenshot{rockbox_interface/images/ss-virtual-keyboard}{The virtual keyboard}{}
This is the virtual keyboard that is used when entering text in Rockbox, for
example when renaming a file or creating a new directory.
The virtual keyboard can be easily changed by making a text file
with the required layout. More information on how to achieve this can be found
on the Rockbox website at \wikilink{LoadableKeyboardLayouts}.

\opt{morse_input}{
  Also you can switch to Morse code input mode by changing the
  \setting{Use Morse Code Input} setting%
  \opt{IRIVER_H100_PAD,IRIVER_H300_PAD,IPOD_4G_PAD,IPOD_3G_PAD,IRIVER_H10_PAD%
      ,GIGABEAT_PAD,GIGABEAT_S_PAD,MROBE100_PAD,SANSA_E200_PAD,PBELL_VIBE500_PAD%
      ,SANSA_FUZEPLUS_PAD,SAMSUNG_YH92X_PAD,SAMSUNG_YH820_PAD}
    { or by pressing \ActionKbdMorseInput{} in the virtual keyboard}%
  .}

% no "Actions" yet in the Player's virtual keyboard

\note{When the cursor is on the input line, \ActionKbdSelect{} deletes the preceding character}

\begin{btnmap}
    \opt{IRIVER_H100_PAD,IRIVER_H300_PAD,GIGABEAT_PAD,GIGABEAT_S_PAD%
        ,MROBE100_PAD,SANSA_E200_PAD,SANSA_FUZE_PAD,SANSA_C200_PAD,SANSA_FUZEPLUS_PAD%
        ,SAMSUNG_YH820_PAD}{
        \ActionKbdCursorLeft{} / \ActionKbdCursorRight
            &
        \opt{HAVEREMOTEKEYMAP}{\ActionRCKbdCursorLeft{} / \ActionRCKbdCursorRight
            &}
        Move the line cursor within the text line.
            \\
        %
        \ActionKbdBackSpace
            &
        \opt{HAVEREMOTEKEYMAP}{
            &}
        Delete the character before the line cursor.
            \\
    }%
    \ActionKbdLeft{} / \ActionKbdRight
        &
    \opt{HAVEREMOTEKEYMAP}{\ActionRCKbdLeft{} / \ActionRCKbdRight
        &}
    Move the cursor on the virtual keyboard.
    If you move out of the picker area, you get the previous/next page of
    characters (if there is more than one).
        \\
    %
    \ActionKbdUp{} / \ActionKbdDown
        &
    \opt{HAVEREMOTEKEYMAP}{\ActionRCKbdUp{} / \ActionRCKbdDown
        &}
    Move the cursor on the virtual keyboard.
    If you move out of the picker area you get to the line edit mode.
        \\
    %
    \nopt{IPOD_3G_PAD,IPOD_4G_PAD,IRIVER_H10_PAD,PBELL_VIBE500_PAD%
         ,SANSA_FUZEPLUS_PAD,SAMSUNG_YH92X_PAD,SAMSUNG_YH820_PAD}{
        \ActionKbdPageFlip
            &
        \opt{HAVEREMOTEKEYMAP}{\ActionRCKbdPageFlip
            &}
        Flip to the next page of characters (if there is more than one).
            \\
    }
    %
    \ActionKbdSelect
        &
    \opt{HAVEREMOTEKEYMAP}{\ActionRCKbdSelect
        &}
    Insert the selected keyboard letter at the current line cursor position.
        \\
    %
    \ActionKbdDone
        &
    \opt{HAVEREMOTEKEYMAP}{\ActionRCKbdDone
        &}
    Exit the virtual keyboard and save any changes.
        \\
    %
    \ActionKbdAbort
        &
    \opt{HAVEREMOTEKEYMAP}{\ActionRCKbdAbort
        &}
    Exit the virtual keyboard without saving any changes.
        \\
% to be done - create a separate section for morse imput and update the info
      \opt{morse_input}{
        \opt{IRIVER_H100_PAD,IRIVER_H300_PAD,GIGABEAT_PAD,GIGABEAT_S_PAD,MROBE100_PADD%
            ,SANSA_E200_PA,IPOD_4G_PAD,IPOD_3G_PAD,IRIVER_H10_PAD,PBELL_VIBE500_PAD%
            ,SAMSUNG_YH92X_PAD,SAMSUNG_YH820_PAD}{
          \ActionKbdMorseInput
          \opt{HAVEREMOTEKEYMAP}{& \ActionRCKbdMorseInput}
          & Toggle keyboard input mode and Morse code input mode. \\}
        %
        \ActionKbdMorseSelect
        \opt{HAVEREMOTEKEYMAP}{& \ActionRCKbdMorseSelect}
        & Tap to select a character in Morse code input mode. \\
      }
\end{btnmap}

% $Id$ %
\section{\label{ref:database}Database}

\subsection{Introduction}
This chapter describes the Rockbox music database system. Using the information
contained in the tags (ID3v1, ID3v2, Vorbis Comments, Apev2, etc.) in your
audio files, Rockbox builds and maintains a database of the music
files on your player and allows you to browse them by Artist, Album, Genre, 
Song Name, etc.  The criteria the database uses to sort the songs can be completely
 customised. More information on how to achieve this can be found on the Rockbox
 website at \wikilink{DataBase}. 

\subsection{Initializing the Database}
The first time you use the database, Rockbox will scan your disk for audio files.
This can take quite a while depending on the number of files on your \dap{}.
This scan happens in the background, so you can choose to return to the
Main Menu and continue to listen to music.
If you shut down your player, the scan will continue next time you turn it on.
After the scan is finished you may be prompted to restart your \dap{} before
you can use the database.

\subsubsection{Ignoring Directories During Database Initialization}

You may have directories on your \dap{} whose contents should not be added
to the database. Placing a file named \fname{database.ignore} in a directory
will exclude the files in that directory and all its subdirectories from
scanning their tags and adding them to the database. This will speed up the
database initialization.

If a subdirectory of an `ignored' directory should still be scanned, place a
file named \fname{database.unignore} in it. The files in that directory and
its subdirectories will be scanned and added to the database.

\subsubsection{Issues During Database Commit}

You may have files on your \dap{} whose contents might not be displayed
correctly or even crash the database.
Placing a file named \fname{/.rockbox/database\_commit.ignore}
will prevent the device from committing the database automatically
you can manually commit the database using the db\_commit plugin in APPS
with logging

\subsection{\label{ref:databasemenu}The Database Menu}

\begin{description}
  \opt{tc_ramcache}{
  \item[Load To RAM]
    The database can either be kept on \disk{} (to save memory), or
    loaded into RAM (for fast browsing). Setting this to \setting{Yes} loads
    the database to RAM, beginning with the next reboot, allowing faster browsing and
    searching. Setting this option to \setting{No} keeps the database on the \disk{},
    meaning slower browsing but it does not use extra RAM and saves some battery on
    boot up.

    \opt{HAVE_DISK_STORAGE}{
      If you browse your music frequently using the database, you should
      load to RAM, as this will reduce the overall battery consumption because
      the disk will not need to spin on each search.
    }

    \note{When Load to RAM is turned on, and the directory cache (see
    \reference{ref:dircache}) is enabled as well, it may take an unexpectedly long amount
    of time for disk activity to wind down after booting, depending on your library size
    and player.

    This can be mitigated by choosing the \setting{Quick} option instead, which causes
    the database to ignore cached file references. In that case, you should expect brief
    moments of disk activity whenever the path for a database entry has to be retrieved.

    Setting this to \setting{On} may be preferable for reducing disk accesses if you plan to
    take advantage of \setting{Auto Update}, have enabled \setting{Gather Runtime Data}
    (see below for both), enabled \setting{Automatic resume} (see
    \reference{ref:Autoresumeconfigactual}), or use a WPS that displays multiple upcoming
    tracks from the current playlist. In the latter case, metadata will not be displayed
    for those tracks otherwise.}
  }

\item[Auto Update]
  If \setting{Auto update} is set to \setting{on}, each time the \dap{}
  boots, the database will automatically be updated.

\item[Initialize Now]
  You can force Rockbox to rescan your disk for tagged files by
  using the \setting{Initialize Now} function in the \setting{Database
    Menu}.
  \warn{\setting{Initialize Now} removes all database files (removing
    runtimedb data also) and rebuilds the database from scratch.}

\item[Update Now]
  \setting{Update now} causes the database to detect new and deleted files
    \note{Unlike the \setting{Auto Update} function, \setting{Update Now}
      will update the database regardless of whether the \setting{Directory Cache}
      is enabled. Thus, an update using \setting{Update now} may take a long
      time.
  }
  Unlike \setting{Initialize Now}, the \setting{Update Now} function
  does not remove runtime database information.
  
\item[Gather Runtime Data]
  When enabled, rockbox will record how often and how long a track is being played, 
  when it was last played and its rating. This information can be displayed in
  the WPS and is used in the database browser to, for example, show the most played, 
  unplayed and most recently played tracks.
  
\item[Export Modifications]
  This allows for the runtime data to be exported to the file \\
  \fname{/.rockbox/database\_changelog.txt}, which backs up the runtime data in
  ASCII format. This is needed when database structures change, because new
  code cannot read old database code. But, all modifications
  exported to ASCII format should be readable by all database versions.
  
\item[Import Modifications.]
  Allows the \fname{/.rockbox/database\_changelog.txt} backup to be 
  conveniently loaded into the database. If \setting{Auto Update} is
  enabled this is performed automatically when the database is initialized.

\item[Select directories to scan.]
  The database normally scans all directories for audio files. This setting
  allows you to limit the scan to a specified list of directories, so only
  files contained in one of these directories will be added to the database.
  Scanning is recursive -- all subdirectories of a selected directory will
  be scanned as well.

\end{description}

\subsection{Using the Database}
Once the database has been initialized, you can browse your music 
by Artist, Album, Genre, Song Name, etc.  To use the database, go to the
 \setting{Main Menu} and select \setting{Database}.\\

\note{You may need to increase the value of the \setting{Max Entries in File
Browser} setting (\setting{Settings $\rightarrow$ General Settings
$\rightarrow$ System $\rightarrow$ Limits}) in order to view long lists of
tracks in the ID3 database browser.\\

There is no option to turn off database completely. If you do not want
to use it just do not do the initial build of the database and do not load it
to RAM.}%

If your total amount of music tracks exceeds the value of the
\setting{Max Playlist Size} setting (\setting{Settings $\rightarrow$ General
Settings $\rightarrow$ System $\rightarrow$ Limits}), using the database
will be your only way to shuffle between all songs from your music library.
Any view on the database browser that exceeds the maximum value of this option
will be automatically adjusted and randomized to fit into the available space
when you will create a dynamic playlist from the view.
Using the database browser is recommended if you shuffle regularly between a lot of
songs rather than increasing your limit, so you will get the best possible performance
on this action.

\note{For your convenience, a shortcut button "Shuffle Songs" is available directly
from the \setting{Database} menu to create and start a mix with all of your
existing music tracks.}

\begin{table}
  \begin{rbtabular}{.75\textwidth}{XXX}%
  {\textbf{Tag}   & \textbf{Type}  & \textbf{Origin}}{}{}
  filename              & string    & system \\ 
  album                 & string    & id tag \\
  albumartist           & string    & id tag \\
  artist                & string    & id tag \\
  comment               & string    & id tag \\
  composer              & string    & id tag \\
  genre                 & string    & id tag \\
  grouping              & string    & id tag \\
  title                 & string    & id tag \\
  bitrate               & numeric   & id tag \\
  discnum               & numeric   & id tag \\
  year                  & numeric   & id tag \\
  tracknum              & numeric   & id tag/filename \\
  autoscore             & numeric   & runtime db \\
  lastplayed            & numeric   & runtime db \\
  playcount             & numeric   & runtime db \\
  Pm (play time -- min)  & numeric   & runtime db \\
  Ps (play time -- sec)  & numeric   & runtime db \\
  rating                & numeric   & runtime db \\
  commitid              & numeric   & system \\
  entryage              & numeric   & system \\
  length                & numeric   & system \\
  Lm (track len -- min)  & numeric   & system \\
  Ls (track len -- sec)  & numeric   & system \\
  \end{rbtabular}
\end{table}


\section{\label{ref:WPSSettings}What's Playing Screen}

  \begin{description}
  \item[Default Browser.]
  Decide if you want the \setting{File Browser}, \setting{Database}, or
  \setting{Playlist Catalogue} to launch after pressing \ActionWpsBrowse{}
  on the WPS, when no other browser has recently been used.
  \opt{hotkey}{
    \item[WPS Hotkey.] Sets the hotkey function for
                     the WPS (see \reference{ref:Hotkeys}). The
                     default is \setting{View Playlist}.
  }
  \item[Set WPS Context Plugin.]
  This option will allow you to run a Rockbox plugin from the WPS context menu.
   \end{description}


%Include playlist section
% $Id$ %
\section{\label{ref:working_with_playlists}Working with Playlists}

\subsection{Playlist terminology}

\begin{description}
\item[Directory.] Rockbox always considers the song that is playing to be part of
  a playlist, and will create a playlist automatically when you are playing the
  contents of a directory. Meaning, just about anything that is described in this
  chapter with respect to playlists also applies to directories.

\item[Dynamic playlist.]  A dynamic playlist is a playlist created
  ``on the fly.'' Any time you use the \setting{Playing Next...} menu
  (see \reference{ref:playingnext_submenu}), or play something
  from the database, you are creating or adding to a dynamic playlist.

\item[Play / Add.] \setting{Play} or \setting{Add} a track to
  put it into the current (dynamic) playlist.

\item[Queue.] \setting{Queued} tracks are also put into the dynamic playlist,
  but removed again as soon as they've been played.
  Note: Options for queuing tracks are hidden by default (see \reference{ref:queuing} for more details).
\end{description}

\subsection{Creating playlists}

\subsubsection{By selecting a song for playback}
When a song is selected from the \setting{File Browser} or \setting{Database} by pressing
\ActionTreeEnter, Rockbox will automatically create a playlist containing all of the
listed songs and will start playback.

\note{Playing a new song will erase the existing dynamic playlist
  and create a new one. If you want to \emph{add} a song to it instead,
  see \reference{ref:playingnext_submenu} on how to choose what's playing next.}

\subsubsection{By choosing ``Play`` or ``Play Shuffled`` from ``Playing Next...``}
Replaces an existing dynamic playlist with the selected tracks.

\subsubsection{\label{ref:addtoplaylist_submenu}By choosing  ``Add to Playlist...``}
Choose \setting{Add to Playlist...} from the \setting{Context Menu} to
add selected track(s) or directory to a new or existing playlist that is not currently
playing.

\note{All playlists in the \setting{Playlist catalogue} are stored by default
  in the \fname{/Playlists} directory in the root of your \daps{} disk and
  playlists stored in other locations are not included in the catalogue. It is
  however possible to move existing playlists there or change the default playlist
  directory (see \reference{ref:Contextmenu}).}

\subsubsection{By using the Main Menu}
To create a playlist containing some or all of the music on your \dap{}, you can use the
\setting{Create Playlist} command in the \setting{Playlist Catalogue Context Menu}
(see \reference{ref:playlistcatalogue_contextmenu}).

\subsection{Choosing what's playing next}

\subsubsection{\label{ref:playingnext_submenu}Adding music to a dynamic playlist}
\screenshot{rockbox_interface/images/ss-playlist-menu}{Playing Next...}{}
\setting{Playing Next...} is a submenu in the \setting{Context Menu} (see
\reference{ref:Contextmenu}) that can be invoked on a selection of tracks in various
places, such as the File Browser, Database, or even PictureFlow:

\begin{description}
\item [Play Next.] Track(s) will play immediately after the currently playing track.

\item [Add.] Add track(s) after the most recently added tracks or, if tracks
have not been added yet, immediately after the currently playing track.

\item [Play Last.] Add track(s) to the end of the playlist.

\item [Add Shuffled.] Add track(s) to the playlist at random positions.

\item [Play Last Shuffled.] Add tracks in a random order to the end of the playlist.
\end{description}

To replace the current dynamic playlist with your selection, choose:

\begin{description}
\item [Play.] Replace all entries in the dynamic playlist with the selected
  tracks. If \setting{Keep Current Track When Replacing Playlist} is set to
  \setting{Yes}, the new tracks will play after the current track finishes
  playing; if no track is playing or the setting is \setting{No}, the new
  tracks will begin playing immediately.

\item [Play Shuffled.] Similar, except the tracks will be added to the new
  playlist in random order.
\end{description}

\label{ref:queuing}The following options are hidden by default, due to their
more complicated behavior. Queued tracks are temporarily added to the dynamic
playlist, but are automatically removed as soon as the tracks have been played.
Queued tracks will not be saved to a playlist file.
A current playlist containing queued tracks can not be bookmarked, even after saving it,
unless you confirm the tracks' removal first (see \reference{ref:createbookmark}).

\begin{description}
\item [Queue Next.] Corresponds to \setting{Play Next}.

\item [Queue.] Corresponds to \setting{Add}.

\item [Queue Last.] Corresponds to \setting{Play Last}.

\item [Queue Shuffled.] Corresponds to \setting{Add Shuffled}.

\item [Queue Last Shuffled.] Corresponds to \setting{Play Last Shuffled}.
\end{description}

\note{Visibility of options to add shuffled tracks or to queue tracks can be toggled by going to
\setting{Settings} $\rightarrow$ \setting{General Settings} $\rightarrow$ \setting{Playlists}
$\rightarrow$ \setting{Current Playlist}. Select either \setting{Show Shuffled Adding Options}
or \setting{Show Queue Options} to customize the displayed set of options.}

If \setting{Playing Next...} is invoked on a directory, Rockbox adds all of the tracks in
that directory to the playlist.

\note{You can control whether or not Rockbox includes the contents of
  subdirectories when adding an entire directory to a playlist. Set the
  \setting{Settings $\rightarrow$ General Settings $\rightarrow$ Playlist
  $\rightarrow$ Recursively Insert Directories} setting to \setting{Yes} if
  you would like Rockbox to include tracks in subdirectories as well as tracks
  in the currently-selected directory.}

Dynamic playlists are saved, so resume will restore them exactly as they
were before shutdown.

\note{To view, save, reshuffle, or display the play time of the current
  dynamic playlist use the
  \setting{Current Playlist} sub menu in the WPS context menu.}

\subsection{Modifying playlists}
\subsubsection{Reshuffling}
Reshuffling the current playlist is easily done from the \setting{Current Playlist}
sub menu in the WPS.

\subsubsection{Moving and removing tracks}
To move or remove a track from the current playlist, enter the
\setting{Playlist Viewer} by selecting \setting{View Current Playlist} in the
\setting{Current Playlist} submenu in the WPS context menu.
Once in the \setting{Playlist Viewer} open the context menu on the track you
want to move or remove. If you want to move the track, select \setting{Move} in
the context menu and then move the blinking cursor to the place where you want
the track to be moved and confirm with \ActionStdOk. To remove a track, simply
select \setting{Remove} in the context menu.

\subsection{Saving playlists}
To save the current playlist, either enter the \setting{Current Playlist} submenu
in the \setting{WPS Context Menu} (see \reference{sec:contextmenu}) and
select \setting{Save Current Playlist}, or enter the context menu for the
\setting{Playlist catalogue} in the \setting{Main Menu} and select
\setting{Save Current Playlist}.
Either method will bring you to the \setting{Virtual Keyboard} (see
\reference{sec:virtual_keyboard}), enter a filename for your playlist and
accept it. If the current playlist contains any queued tracks, you will be
asked whether to remove them, as a prerequisiste for creating bookmarks
(see \reference{ref:createbookmark}).

\subsection{Loading saved playlists}
\subsubsection{Through the \setting{File Browser}}
Playlist files, like regular music tracks, can be selected through the
\setting{File Browser}. When loading a playlist from disk it will replace
the current dynamic playlist. If you want to look at a playlist's
content without starting playback immediately, access the \setting{Context Menu} (see
\reference{ref:Contextmenu}) with \ActionStdContext{} and choose \setting{View}.

\subsubsection{Through the \setting{Playlist catalogue}}
The \setting{Playlist catalogue} offers a shortcut to all playlists in your
\daps{} specified playlist directory.
It can be used like the \setting{File Browser} but will display
the content of a playlist when one is selected.


% $Id$ %
\opt{hotkey}{
    \section{\label{ref:Hotkeys}Hotkeys}
    Hotkeys are shortcut keys for use in the \nopt{touchscreen}{\setting{File Browser},
    \setting{Database}, \setting{Playlist Viewer}, and }\setting{WPS} screen.  To use one, press
    \nopt{touchscreen}{\ActionTreeHotkey{} within the \setting{File Browser},
    \setting{Database}, or \setting{Playlist Viewer}, or}
    \ActionWpsHotkey{} within the \setting{WPS}
    screen.\nopt{touchscreen}{ The assigned function will launch with reference
    to the current file or directory, if applicable.  Each screen has its own
    assignment.} If there is no assignment for a given screen,
    the hotkey is ignored.

    The hotkey assignments are changed for the What's Playing Screen (see
    \reference{ref:WPSSettings}) and browsers (see \reference{ref:FileView})
    separately.

    The default assignment for the \nopt{touchscreen}{File Browser hotkey is
    \setting{Off}, while the default for the }WPS hotkey is
    \setting{View Playlist}.
}

